\chapter{绪论}
\section{引言}
近年来,随着工业化的不断发展,工业以及生活中的含油废污水也日益增多。由其引起的水资源短缺和环境污染问题也日益严峻,严重制约着世界经济的可持续发展。工业以及生活过程中排放的各类含油污水通过多种形式严重危害自然环境和人类健康。油水混合物的分离以及污水的纯化再利用逐渐成为全世界范围的研究课题。

在当今社会的工业生产活动中,众多涉及油类的工业和生活活动中都会产生大量的含油废水。例如石油行业中,从开采到提纯,最后到运输,都会产生含油废水以及油水分离的问题。石油的开采过程中,为了保证油层压力,多采用油田注水的方式来获得较高的采收率,保证油田高产,产生了成分复杂的含油废水。在石油提纯和分离时,要去除油中的水,也需要一种高效的油水分离方式。另外,海上钻井平台的泄漏以及油轮运输也会造成极大的污染问题。1989年3月24日,美国埃克森公司的一艘巨型油轮在阿拉斯加州的威廉王子湾附近触礁,原油泄出达800多万加仑,在海面上形成一条宽约1公里、长达800公里的漂油带,附近的生态环境和渔业都遭到极大的破坏。除了石油行业,在纺织,皮革,食品加工等行业中也会产生大量的含油废水,含油废水的处理不当,更是会加剧环境污染问题。所以,油水分离,不论是从环保角度还是从工业角度,一直以来都是一个必须解决的问题,而且也是一个充满挑战困难重重的课题。由于油水含量,混合状态、混合物的复杂组分、油水分离对象等多种不同的因素。导致了油水分离方式的多种多样。

传统的污水处理方式有重力沉降法、撇油法、凝固法、过滤法、生物法等。传统分离技术。这些传统的含油污水处理方法通常都有耗时长,耗能大,并且设备体积庞大,占地面积大,并且效率低,需要组合搭配使用等缺点。在众多的污水分离技术当中,膜处理技术由于其出色的分离效率,以及结构紧凑、能耗低、操作方便,适用范围广,不产生二次污染等优点,逐渐成为最广泛应用的废污水处理方式。

膜材料是膜分离技术的核心,其已经从最初的纤维素衍生物类,逐步发展为无机陶瓷、金属、高分子等多种材料。但传统的膜分离技术中,尤其是对于高分子膜,膜的污染问题是一个难以避免的问题。膜的污染问题主要源自于传统分离膜的表面疏水性,膜的表面容易受到油的黏附污染从而会造成膜的通量的快速下降,油水分离效果也会大打折扣,进而降低膜的使用寿命,影响分离膜的广泛应用。目前的研究,众多研究学者采用提高高分子滤膜的亲水性来增加膜的抗污染性能。\par 本章主要是介绍(1)超浸润分离膜的研究进展以及应用(2)超浸润分离膜的亲水改性研究进展。

测试脚注\footnote{分别编号}。

\section{油水分离的进展}

\subsection{油水污染的产生及危害}
在全球化的工业化进程中,石化、机械、纺织、皮革和金属加工等工业生产过程中,会产生大量的含邮废污水,这些废污水直接排放到大自然中,会以多种方式危害自然环境和人体健康,并且会进一步加重水资源短缺的危机。\par 
通过油水分离的方式处理水中的油污染物,已经被广泛的认为是一种可以持续发展的策略,油水分离已经成为全球研究的热点问题,而高性能膜材料的研发和发展已经成为我国的十三五专项规划之一。
\subsection{海洋油污的处理}
广袤的海洋占了地球表面积的70\%以上,有着丰富多样的生物资源,而且蕴含着难以估量的矿产资源,然而,在开发利用海洋资源的过程中,船舶、采油平台等设施却不可避免地造成海洋油污的问题,如海上油气钻井平台的泄露,油轮的溢油等。油类溢漏到海洋中,会在水面迅速扩散,并会隔离海水与空气中的氧气交换。更重要的是,部分油污还能溶解于水中,这样的含油污水1升完全氧化,需要耗掉40万升水中的氧,导致大量的海洋生物的死亡。海洋油污问题严重阻碍着海洋的生态环境,每年由海洋油污造成的损失难以估算,因此,海洋油污处理已经逐渐引起世界各国的重视。海洋油污事故具有复杂性、突发性、社会性、长期性等特点。快速的处理海洋油污事故就能尽快的降低损失,所以就要研发一种快捷高效的油水分离方式。\par
传统的海洋油污处理方式有物理方法(围油栏、撇油器、吸油毡)、化学方法(溢油分散剂、凝固剂、就地燃烧)和生物方法(微生物)。其中化学方法和生物方法容易造成对海洋环境造成二次污染。撇油器等机械方法收集的溢油为油水混合物,后期处理难度比较大。活性炭和吸油毡等材料可通过物理吸附作用实现油水分离,但其吸油速率慢,重复利用性差。传统分离膜上附着一些超浸润,对原油有低黏附力的涂层,是一种新型的海洋溢油快速处理的办法。\par 
宁波材料所的科研人员通过分子刷技术,研究出一种高强度、可重复使用的吸油疏水的三维多孔弹性高分子材料。
\subsection{油水分离技术的发展发现}
具有特殊表面润湿性的复合材料可以实现有效的油水分离功能。

\section{膜分离技术简介}
\subsection{膜的种类和制膜方法}
\subsection{膜分离技术发展情况和瓶颈}
\subsection{膜的改性方法}
\subsubtion{表面改性}
\subsubtion{物理改下}
\subsubsubtion{接枝改性}
\subsubsubtion{共混改性}
\section{PVDF膜的研究进展}

\section{论文的选题背景和主要研究内容}

测试脚注\footnote{脚注2}
